% This is ''sig-alternate.tex'' V2.0 May 2012
% This file should be compiled with V2.5 of '\'sig-alternate.cls'' May 2012
%
% This example file demonstrates the use of the \'sig-alternate.cls'
% V2.5 LaTeX2e document class file. It is for those submitting
% articles to ACM Conference Proceedings WHO DO NOT WISH TO
% STRICTLY ADHERE TO THE SIGS (PUBS-BOARD-ENDORSED) STYLE.
% The \'sig-alternate.cls' file will produce a similar-looking,
% albeit, 'tighter' paper resulting in, invariably, fewer pages.

\documentclass{sig-alternate}
\sloppy
\usepackage{paralist}
\usepackage{url}
\def\UrlBreaks{\do\/\do-}
\usepackage[pdftex,breaklinks]{hyperref}
\usepackage{import}
\usepackage{tikz}
%\usepackage{caption}
%\captionsetup[figure]{skip=0pt}

\begin{document}
%
% --- Author Metadata here ---
\conferenceinfo{SIGCSE}{2017 Seattle, Washington, USA}
\CopyrightYear{2017} % Allows default copyright year (20XX) to be over-ridden - IF NEED BE.
%\crdata{0-12345-67-8/90/01}  % Allows default copyright data (0-89791-88-6/97/05) to be over-ridden - IF NEED BE.
% --- End of Author Metadata ---

\title{An Analysis of Introductory University Programming Courses in the UK}

\numberofauthors{3}
\author{
% 1st. author
\alignauthor
Ellen Murphy\\
\affaddr{Institute for\\Mathematical Innovation}\\
\affaddr{University of Bath}\\
\affaddr{e.murphy@bath.ac.uk}
% 2nd. author
\alignauthor
Tom Crick\\
\affaddr{Dept. of Computing}\\
\affaddr{Cardiff Metropolitan University}\\
\affaddr{tcrick@cardiffmet.ac.uk}
% 3rd. author
\alignauthor
James H. Davenport\\
\affaddr{Dept. of Computer Science}\\
\affaddr{University of Bath}\\
\affaddr{j.h.davenport@bath.ac.uk}\\
}

\maketitle

\begin{abstract}
This paper reports the results of a survey of xx introductory
programming courses delivered at UK universities as part of their
first year computer science (or similar) degree programmes, conducted
in the first half of 2016. Results of this survey are compared with a
related survey conducted since 2010 (as well as earlier surveys from
2001 and 2003) on universities in Australia and New Zealand. Trends in
student numbers, programming paradigm, programming languages and
environment/tools used, as well as the reasons for choice of such are
reported. Other aspects of first programming courses such as
instructor experience, external delivery of courses and resources
given to students are also examined.

The results indicate a trend towards...
\end{abstract}

% check these...
% A category with the (minimum) three required fields
\category{K.3.2}{Computers \& Education}{Computer and Information Science Education}[Computer Science Education]
\category{K.4.1}{Computers And Society}{Public Policy Issues}
\keywords{Introductory Programming, Programming Languages, Programming
  Environments, Computer Science Education, Higher Education, Tertiary
  Education, UK}

\section{Introduction}\label{intro}

\cite{mason+cooper:2014} is the latest in a long line \cite{deraadt-et-al:2004,mason-et-al:2012} of papers surveying the teaching of introductory programming courses in Australasia. However, such surveys are not the norm elsewhere, and this paper reports the authors' findings from runningthe first such survey in the United Kingdom.

UK policy context
e.g. schools~\cite{brown-et-al-sigcse2013,brown-et-al-toce2014},
Shadbolt/Wakeham, TEF, graduate employability, etc.

Other work in this
space~\cite{mccracken-et-al:2001,gupta:2004,dale:2006,pears-et-al:2007,guo:2014}

Also, our previous work~\cite{crick-et-al-hea:2015,davenport-et-al:latice2016}.

%Hat tip to the Australian work~\cite{deraadt-et-al:2004,mason-et-al:2012,mason+cooper:2014}.

\section{Methodology}\label{method}

\subsection{Recruitment of Participants}

\subsection{Questions}


\section{Results and Discussion}\label{results}

\subsection{Universities and Courses}

\begin{figure}
\begin{center}
\subimport{plots/}{tariffGroup.tex}
\caption{The number of universities per Tariff Group.}
\end{center}
\end{figure}

\subsection{Student Numbers}

\subsection{Languages}

\begin{table}[]
\centering
\caption{The number of programming languages used in first programming courses.}
\label{tab:numLanguages}
\begin{tabular}{ccccc}
\hline
Languages & 1  & 2  & 3 & 4 \\ \hline
Courses   & 61 & 17 & 3 & 1 \\ \hline
\end{tabular}
\end{table}

\begin{figure}
\begin{center}
\subimport{plots/}{langPercentCompare.tex}
\end{center}
\caption{Language popularity by percentage of courses and students.}
\end{figure}

\begin{figure}
\begin{center}
\subimport{plots/}{reasonsByCourseCompare.tex}
\end{center}
\caption{Reasons given for choosing a programming language by percentage for: all languages; Java; and Python.}
\end{figure}


\begin{figure}
\begin{center}
\subimport{plots/}{UseAndDifficultyCompareLanguages.tex}
\end{center}
\caption{The median of the perceived difficulty and usefulness of language, where 1 is `extremely easy' and 7 is `extremely difficult' for difficulty and 1 is `extremely useful' and 7 is `extremely useless' for usefulness.  Answers must have been given by at least two instructors.}
\end{figure}




\begin{table}[]
\centering
\caption{The main paradigm in use in the first programming course.}
\label{tab:paradigm}
\begin{tabular}{cccc}
\hline
Paradigm & Object-Oriented & Procedural & Functional \\ \hline
Courses  & 40              & 27         & 7          \\ \hline
\end{tabular}
\end{table}


\subsection{Instructor Experience}

\begin{table}[]
\centering
\caption{The number of years the instructor has been involved in teaching introductory programming.}
\label{tab:yearsTeaching}
\begin{tabular}{ccccccc}
\hline
Years       & \textless 2 & 2 - 5 & 5 - 10 & 10 - 20 & 20 - 30 & \textgreater 30 \\ \hline
Instructors & 3          & 9     & 9      & 27      & 19      & 7              \\ \hline
\end{tabular}
\end{table}

\subsection{IDEs and Tools}

\begin{table}[]
\centering
\caption{The number of tools/environments used in first programming courses.}
\label{tab:numTools}
\begin{tabular}{cccccc}
\hline
Tools   & 1  & 2  & 3 & 4 & 8 \\ \hline
Courses & 34 & 15 & 6 & 2 & 1 \\ \hline
\end{tabular}
\end{table}

\begin{figure}
\begin{center}
\subimport{plots/}{toolPercentCompare.tex}
\caption{Tool or environment popularity by percentage of courses and students.}
\end{center}
\end{figure}

\begin{figure}
\begin{center}
\subimport{plots/}{reasonsByCourseCompareTool.tex}
\end{center}
\caption{Reasons given for choosing a tool or environment by percentage for: all tools and environments; BlueJ; and Eclipse.}
\end{figure}

\begin{figure}
\begin{center}
\subimport{plots/}{timingOtherCourseTool.tex}
\end{center}
\caption{For each tool or environment, whether it is used: for an initial part of the first programming course; throughout the whole of the first programming course; in any other course in the degree.}
\end{figure}


\begin{figure}
\begin{center}
\subimport{plots/}{DifficultyYouStudentsCompareTools.tex}
\end{center}
\caption{The median difficulty rating of tool/environment for the instructor and students to use, where 1 is `extremely easy' and 7 is `extremely difficult'.  Answers must have been given by at least two instructors.}
\end{figure}

\subsection{Other Aspects of the Course}

\begin{figure}
\begin{center}
\subimport{plots/}{Steps.tex}
\end{center}
\caption{Steps taken to determine whether students have received unauthorised assistance on assignments.}
\end{figure}

\begin{figure}
\begin{center}
\subimport{plots/}{Resources.tex}
\end{center}
\caption{Resources provided to students.}
\end{figure}
\subsection{Aims of an Introductory Programming Course}


\section{General Discussion}\label{discussion}

\subsection{The U.K. context}

\subsection{Comparison with Australasia}
Here we compare with \cite{mason+cooper:2014}, the latest Australasian survey.


\section{Acknowledgements}

The authors would like to thank the participants for their engagement
with the survey, as well as...
% HEFCE?
% CPHC?
% GW4 - JHD has asked what the form of words should be.

% bib
\bibliographystyle{abbrv}
\bibliography{sigcse2017}

\end{document}
