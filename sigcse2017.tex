% This is ''sig-alternate.tex'' V2.0 May 2012
% This file should be compiled with V2.5 of '\'sig-alternate.cls'' May 2012
%
% This example file demonstrates the use of the \'sig-alternate.cls'
% V2.5 LaTeX2e document class file. It is for those submitting
% articles to ACM Conference Proceedings WHO DO NOT WISH TO
% STRICTLY ADHERE TO THE SIGS (PUBS-BOARD-ENDORSED) STYLE.
% The \'sig-alternate.cls' file will produce a similar-looking,
% albeit, 'tighter' paper resulting in, invariably, fewer pages.

\documentclass{sig-alternate}
\sloppy
\usepackage{paralist}
\usepackage{url}
\usepackage[pdftex]{hyperref}

\begin{document}
%
% --- Author Metadata here ---
\conferenceinfo{SIGCSE}{2017 Seattle, Washington, USA}
\CopyrightYear{2017} % Allows default copyright year (20XX) to be over-ridden - IF NEED BE.
%\crdata{0-12345-67-8/90/01}  % Allows default copyright data (0-89791-88-6/97/05) to be over-ridden - IF NEED BE.
% --- End of Author Metadata ---

\title{An Analysis of Introductory University Programming Courses in the UK}

\numberofauthors{3}
\author{
% 1st. author
\alignauthor
Ellen Murphy\\
\affaddr{Institute for Mathematical Innovation}\\
\affaddr{University of Bath, UK}\\
\affaddr{e.murphy@bath.ac.uk}
% 2nd. author
\alignauthor
Tom Crick\\
\affaddr{Department of Computing}\\
\affaddr{Cardiff Metropolitan University, UK}\\
\affaddr{tcrick@cardiffmet.ac.uk}
% 3rd. author
\alignauthor
James H. Davenport\\
\affaddr{Department of Computer Science}\\
\affaddr{University of Bath, UK}\\
\affaddr{j.h.davenport@bath.ac.uk}\\
}

\maketitle

\begin{abstract}
This paper reports the results of a survey of xx introductory
programming courses delivered at UK universities as part of their
first year computer science (or similar) degree programmes, conducted
in the first half of 2016. Results of this survey are compared with a
related survey conducted since 2010 (as well as earlier surveys from
2001 and 2003) on universities in Australia and New Zealand. Trends in
student numbers, programming paradigm, programming languages and
environment/tools used, as well as the reasons for choice of such are
reported. Other aspects of first programming courses such as
instructor experience, external delivery of courses and resources
given to students are also examined.

The results indicate a trend towards...
\end{abstract}

% check these...
% A category with the (minimum) three required fields
\category{K.3.2}{Computers \& Education}{Computer and Information Science Education}[Computer Science Education]
\category{K.4.1}{Computers And Society}{Public Policy Issues}
\keywords{Introductory Programming, Programming Languages, Programming
  Environments, Computer Science Education, Higher Education, UK}


\section{Introduction}\label{intro}

UK policy context e.g. schools~\cite{brown-et-al-sigcse2013,brown-et-al-toce2014}.


\section{Methodology}\label{method}


\section{Results and Discussion}\label{results}


\section{General Discussion}\label{discussion}


\section{Acknowledgements}

% bib
\bibliographystyle{abbrv}
\bibliography{sigcse2017}

\end{document}
