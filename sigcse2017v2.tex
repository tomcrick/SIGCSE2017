% This is ''sig-alternate.tex'' V2.0 May 2012
% This file should be compiled with V2.5 of '\'sig-alternate.cls'' May 2012
%
% This example file demonstrates the use of the \'sig-alternate.cls'
% V2.5 LaTeX2e document class file. It is for those submitting
% articles to ACM Conference Proceedings WHO DO NOT WISH TO
% STRICTLY ADHERE TO THE SIGS (PUBS-BOARD-ENDORSED) STYLE.
% The \'sig-alternate.cls' file will produce a similar-looking,
% albeit, 'tighter' paper resulting in, invariably, fewer pages.

\documentclass{sig-alternate}
\sloppy
\usepackage{paralist}
\usepackage{url}
\def\UrlBreaks{\do\/\do-}
\usepackage[pdftex,breaklinks]{hyperref}
\usepackage{import}
\usepackage{tikz}
%\usepackage{caption}
%\captionsetup[figure]{skip=0pt}

\begin{document}
%
% --- Author Metadata here ---
\conferenceinfo{SIGCSE}{2017 Seattle, Washington, USA}
\CopyrightYear{2017} % Allows default copyright year (20XX) to be over-ridden - IF NEED BE.
%\crdata{0-12345-67-8/90/01}  % Allows default copyright data (0-89791-88-6/97/05) to be over-ridden - IF NEED BE.
% --- End of Author Metadata ---

\title{An Analysis of Introductory University Programming Courses in the UK}
\iffalse
\numberofauthors{3}
\author{
% 1st. author
\alignauthor
Ellen Murphy\\
\affaddr{Institute for\\Mathematical Innovation}\\
\affaddr{University of Bath}\\
\affaddr{e.murphy@bath.ac.uk}
% 2nd. author
\alignauthor
Tom Crick\\
\affaddr{Dept. of Computing}\\
\affaddr{Cardiff Metropolitan University}\\
\affaddr{tcrick@cardiffmet.ac.uk}
% 3rd. author
\alignauthor
James H. Davenport\\
\affaddr{Dept. of Computer Science}\\
\affaddr{University of Bath}\\
\affaddr{j.h.davenport@bath.ac.uk}\\
}
\fi
\maketitle

\begin{abstract}
This paper reports the results of a survey of 80 introductory
programming courses delivered at UK universities as part of their
first year computer science (or similar) degree programmes, conducted
in the first half of 2016. Results of this survey are compared with a
related survey conducted since 2010 (as well as earlier surveys from
2001 and 2003)  in Australia and New Zealand. % Trends {\bf{[don't think we can say trend as only one time point]}} in %JHD: agreed
We report on 
student numbers, programming paradigm, programming languages and
environment/tools used, as well as the reasons for choice of such.

% are reported. Other aspects of first programming courses such as instructor experience, external delivery of courses and resources given to students are also examined.

The results in this first UK survey indicate a trend towards...

...especially in the context of substantial computer science
curriculum reform in UK schools, as well as increasingly scrutiny of
teaching excellence and graduate employability for UK universities.
\end{abstract}

% check these...do we need keywords too?
% A category with the (minimum) three required fields
\category{K.3.2}{Computers \& Education}{Computer and Information Science Education}[Computer Science Education]
\category{K.4.1}{Computers And Society}{Public Policy Issues}
\keywords{Introductory Programming, Programming Languages, Programming
  Environments, Computer Science Education, Higher Education, Tertiary
  Education, UK}

\section{Introduction}\label{intro}

For many years -- and increasingly at all levels of compulsory and
post-compulsory education -- the choice of programming language to
introduce basic programming principles, constructs, syntax and
semantics has been regularly revisited. Even in the context of what
are perceived to be the most difficult introductory topics in computer
science degrees, numerous key themes across programming
appear~\cite{dale:2006}. 

So what is a good first programming language? The issues surrounding
choosing a first language~\cite{gupta:2004,kaplan:2010} -- and a
search of the ACM Digital Library identified a number of papers of the
form ``{\emph{X as a first programming language}}'', going as far back
as the 1980s -- appear to be legion, along with the potential impact
on students' grades and attainment~\cite{ivanovic-et-al:2015}. Decades
of research on the teaching of introductory programming has had
limited effect on classroom practice~\cite{pears-et-al:2007}; although
relevant research exists across several disciplines including
education and cognitive science, disciplinary differences have made
this material inaccessible to many computing educators. Furthermore,
computer science instructors have not had access to comprehensive
surveys of research in this
area~\cite{mccracken-et-al:2001,pears-et-al:2007}.

However, in Australia and New Zealand there have been longitudinal
data
collections~\cite{deraadt-et-al:2004,mason-et-al:2012,mason+cooper:2014}
surveying the teaching of introductory programming courses in
universities. Surprisingly, such surveys have not been conducted
elsewhere on this scale, and this paper reports the findings from
running the first such similar survey in the UK.

We remind the reader that the UK consists of four nations (with an
overall population of 64.5 million: England: 54.3 million, Scotland:
5.3 million, Wales: 3.1 million and Northern Ireland: 1.8 million)
historically ruled by one parliament, but now with devolved assemblies
all responsible for four separate education systems. In the context of
increasing international focus on curriculum and qualification reform
to support computer science education and digital skills in schools,
the four education systems of the UK have proposed and implemented a
variety of changes~\cite{rs:2012,brown-et-al-toce2014}, particular in
England, with a new compulsory computing curriculum for ages 5-16 from
September 2014. For universities across the UK offering computer
science degrees, this curriculum reform in schools has had uncertain
(and emerging) impact on their undergraduate programmes, with the
diversity of educational background of applicants likely to be
increasing before it narrows: it is certainly possible for prospective
students to have anywhere from zero to four or five years experience
(and potentially two school qualifications) in computer science with
some knowledge of programming.

% In 1997, Scotland and Wales held referenda which determined in both
% cases the desire for self-government (along with Northern Ireland and
% the 1998 Good Friday Agreement), creating assemblies to which a
% variety of powers -- in particular, education -- were devolved from
% the UK Parliament.

Over the past few years, there has been increased scrutiny of the
quality of teaching in UK universities, partly linked to the current
levels -- and potential future increases -- of tuition fees (generally
paid by the student through government-supported loans), as well as
the perceived value of professional body accreditation and graduate
employability, especially for STEM disciplines. In February 2015, the
UK Department of Business, Innovation \& Skills initiated independent
reviews of STEM degree accreditation and graduate
employability\footnote{\url{https://www.gov.uk/government/collections/graduate-employment-and-accreditation-in-stem-independent-reviews}},
with a specific review -- the Shadbolt review~\cite{shadbolt:2016} --
focusing on computer science degree accreditation (in this case, with
BCS, The Chartered Institute for IT) and graduate employability,
reporting back in May 2016. Alongside a number of recommendations to
address the relatively high unemployment rates of computer sciences
graduates, particular on quality of data, course types, gender and
demographics, the Shadbolt review split generalist universities in
England into three bands, based on their average (across all subjects)
entrance tariff (qualifications of entrants); we have followed that
banding during our analysis the English results, so our data should
allow comparisons.

\iffalse
% should we mention TEF?
Alongside this increased scrutiny of standards and outputs for
computer science degrees in UK universities, a Teaching Excellence
Framework\footnote{\url{http://www.hefce.ac.uk/lt/tef/}} has been
proposed as part of proposed new higher education legislation. The
core ambition of the framework is ``to raise the quality and status of
teaching in higher education institutions''; excellence is to be
measured through a series of proxy metrics that include student
satisfaction, retention and graduate employability. There has been
significant concerns about the aims of the framework, as well as the
suitability of the metrics; more so in the context of it benchmarking
and creating leagues tables for ``teaching excellence'', as well as
deciding whether institutions are allowed to raise tuition fees in the
future. The UK's Higher Education Academy -- the national body which
champions teaching quality -- has previously supported initiatives for
improving learning \& teaching in computer science, including
innovative pedagogies for
programming~\cite{crick-et-al-hea:2015,davenport-et-al:latice2016},
but we have not seen the desired cascading of best practice and wider
national impact.
% should we mention HEA and our work?
\fi 

In this emerging environment of policy, curricula, pedagogy and the
evolving demands of high-quality learning \& teaching for computer
science degree programmes, we present the findings from the first
national scale survey of introductory programming languages at UK
universities. Through this survey, we identify and analyse trends in
student numbers, programming paradigm, programming languages and
environment/tools used, as well as the reasons for choice of such are
reported. % Other aspects of first programming courses such as
instructor experience, %external delivery of courses: there was none, so irrelevant
%and resources given to students are also examined, along with comparisons to the  Australasian surveys.

\begin{figure}
\begin{center}
\subimport{plots/}{tariffGroupCompare.tex}\vskip-12pt
\caption{The number of responding universities per Nation/   
 Tariff Group.\label{fig:TG}}
\end{center}
\end{figure}

%\subsection{Student Numbers}
%??
\begin{table}[]
\centering
\caption{The number of programming languages used in first programming courses.\label{tab:numLangs}}
\label{tab:numLanguages}
\begin{tabular}{ccccc}
\hline
Languages & 1  & 2  & 3 & 4 \\ \hline
Courses   & 59 & 17 & 3 & 1 \\ \hline
\end{tabular}
\end{table}
\section{Methodology}\label{method}

\subsection{Recruitment of Participants}

To recruit for the survey, a general call for participants was sent
out to the Council of Professors and Heads of Computing (CPHC)
membership; CPHC is the representative body for university computer
science departments in the UK, with nearly 800 members at over 100
institutions\footnote{\url{https://cphc.ac.uk/who-we-are/}}. The
survey was hosted online and was available from mid-May until the end
of June 2016; the invitation asked for the survey to be passed on to
the most appropriate person for that institution to complete it. Due
to the recruitment method, there were a number of duplicate responses
from certain departments, and these were reconciled by direct enquiry.

\begin{figure}
\begin{center}
\subimport{plots/}{langPercentCompare.tex}
\end{center}\vskip-12pt
\caption{Language popularity by percentage of courses and students (excl. OU).\label{fig:lang}}
\end{figure}

\subsection{Questions}

The questions used in the survey were generously provided by the
authors of ~\cite{mason+cooper:2014}, so as to allow direct comparison
between the results of this survey and that of the 2014 Aus/NZ
survey. Where possible, questions were left unchanged, although a
small minority were edited to reflect the UK target audience. As
defined in the 2014 Aus/NZ survey~\cite{mason+cooper:2014}, the terminology
``course'' was used for ``the basic unit of study that is completed by
students towards a degree, usually studied over a period pf a semester
or session, in conjunction with other units of study''.

The first section of the survey asked about the programming
language(s) in use, the reasons for their choice, and their perceived
difficulty and usefulness. Then, questions regarding the use of
environments or development tools; which ones were used, the reasons
for their choice and the perceived difficulty. General questions about
paradigm, instructor experience and external delivery were asked,
along with questions regarding students receiving unauthorised
assistance, and the resources provided to students. Finally,
participants were asked to identify their top three main aims when
teaching introductory programming, and were also allowed to provide
further comments.

% @James: I checked the paper - they could rank all reasons, not just top three.
In the 2014 Aus/NZ survey~\cite{mason+cooper:2014}, participants were
asked to rank the importance of the given reasons for choosing a
programming language, environment or tool. Due to technical
limitations in the online survey tool used, it was not possible to do
so in this survey, so Figure~\ref{fig:reasons} just reports
counts. Most questions were not mandatory; the exceptions were ``what
programming language(s) are in use?'' and a small number of feeder
questions to allow the survey to function correctly.

\section{Results and Discussion}\label{results}

\subsection{Universities and Courses}
%JHD: we need some discussion of response rates, either here or via
%Figure 1 also showing response rates, as in e-mail to Ellen

Upon completion of the survey, 155 instructors had, at least, started
the survey. Sixty-one of these dropped out before answering the
mandatory questions, and a further 14 were duplicates. Therefore, the
results presented here are drawn from the responses of 80 instructors
from at least 70 universities. Some participants did not answer all
questions and due to this the response rate varies by question.


\begin{figure}
\begin{center}
\subimport{plots/}{reasonsByCourseCompare.tex}
\end{center}\vskip-18pt
\caption{Reasons given for choosing a programming language by percentage for: all languages; Java; and Python.\label{fig:reasons}}
\end{figure}
\par
Excluding the Open University's 3200 students, the participants in the
survey represented 13462 students, with a mean of 173 (but a standard
deviation of 88). Looking at Figure \ref{fig:TG} we see good response
rates, apart from the specialist HEIs (most of whom do not teach
computing) and the ``low tariff'' English ones. Fewer of these teach
computing, % e.g. all the 'arts' ones but we are not convinced that
this factor alone explains the response rate. In Northern Ireland, we
had responses from the two universities, but not the university
colleges, which are historically teacher-training colleges.

\subsection{Languages}

\subsubsection{Choice of Language(s)}

One of the mandatory questions in the survey, and a major point of interest related to the programming languages in use in introductory programming courses. Participants were asked to select languages from a list of 22 programming languages and also had the option to choose ``Other'' and specify a language not included in the list. As shown in Table~\ref{tab:numLangs}, the majority of courses surveyed (59 out of 80) use only one programming language. From the 80 courses, the total number of {\emph{language instances}} is 106, as some courses use more than one language to teach introductory programming. 


Of the 22 languages provided, 13 were selected at least once. The relative popularity of languages is shown in Figure~\ref{fig:lang}, where the prevalence is given by the percentage of a language over all language instances (106 total), and weighted by student numbers (16662 total). 
%The programming languages that were not selected at all were: Actionscript, Ada, Delphi, Eiffel, Fortran, jBase, Lisp, Ruby and Visual Basic.

With regard to the time a language is used, of 93 language instances, the majority (65\%) are used for the whole of the introductory programming course, 14.0\% of language instances are used in the first part of a course and 21.5\% of language instances are used after another programming language.

\begin{figure}[ht]
\begin{center}
\subimport{plots/}{langByTariffPercent.tex}
\end{center}%\vskip-12pt  JHD: Oddly, this vskip seems counterproductive
%\caption{The breakdown of programming languages for each of the Tariff Groups.}
%\end{figure}
%
%\begin{figure}
\begin{center}
\subimport{plots/}{TariffByLangPercent.tex}
\end{center}\vskip-18pt
\caption{The breakdown of programming languages by Nation and Tariff Groups.\label{fig;LangTariff}}
\end{figure}

The relative popularity of languages is the immediate major difference with
\cite{mason+cooper:2014}. Their survey showed a dead heat (27.3\% of
language instances) between Java and Python, with Python winning (33.7\% to
26.9\%) when weighted by the number of students on the course.  Our
findings (Figure \ref{fig:lang}) show that Java is a clear winner by
any metric, being used in over half the courses (61.3\%) and just under half of all language instances (46.2\%), while the
runner-up, Python, is in use in 17.5\% of courses and makes up 13.2\% of language instances. The C family (C, C++ and C\#) together
is in use in 31.3\% of introductory programming courses, and scores 23.6\% of language instances and 19.5\% by students. Figure
\ref{fig:lang} shows the effect of student-number weighting \emph{but}
we have excluded the Open University from this weighting, as its 3200
students learning Python (and Sense, a variant of Scratch) would have
distorted the comparison.

% ~\cite{guo:2014} -- US universities and Python being the most popular

\subsubsection{Reasons for choice of language}
For each language selected, participants were asked to give the reasons for choosing that language for the introductory programming course. Figure~\ref{fig:reasons} shows the frequency of these reasons for all languages grouped together and for Java and Python individually. When the reasons given are combined for all languages, three reasons tie for first place: ``relevance to industry''; ``object-oriented language''; and ``availability and cost to students'', all chosen by 54.5\% of participants who answered this question.

Looked at individually, the most popular reason given for choosing Java is ``object-oriented language'' at 87.2\%, while Python scores highest on ``pedagogical benefits'', at 72.7\%. This may explain the popularity of Java: Java scores higher on ``relevance to industry'' and, perhaps somewhat surprisingly, much higher on ``object-oriented language'' than Python, which scores only 18.2\%.
% @Tom: do you agree with ``somewhat surprisingly''?

\subsubsection{Language choice by Nation and Tariff Score}

Figure~\ref{fig;LangTariff} breaks down the choice of language by
nation and tariff group.  It is noticeable that the three English
tariff groups differ significantly, with Python outnumbering Java in
the low tariff universities, and C being almost exclusively in the
high tariff universities.


\subsubsection{Perceived difficulty and usefulness to fundamental concepts}
{\bf{@James: Need to say something about Fig 4: difficulty and usefulness.}}



\begin{figure}
\begin{center}
\subimport{plots/}{UseAndDifficultyCompareLanguages.tex}
\end{center}\vskip-18pt
%\caption{The median of the perceived difficulty and usefulness of language, where 1 is `extremely easy' and 7 is `extremely difficult' for difficulty and 1 is `extremely useful' and 7 is `extremely useless' for usefulness.  Answers must have been given by at least two instructors.}
\caption{The median of the perceived difficulty and [pedagogic] usefulness of language, where 1 is `extremely easy/useful' and 7 is `extremely difficult/useless'.  Answers must have been given by at least two instructors.\label{fig:utility}}
\end{figure}
\begin{table}[ht]
\centering
\caption{The main paradigm in use in the first programming course.}
\label{tab:paradigm}
\begin{tabular}{ccccc}
\hline
Paradigm & OO & Procedural & Functional & No Answer \\ \hline
Courses  & 40              & 27         & 7    & 6      \\ \hline
\end{tabular}
\end{table}
\subsection{Paradigm taught}

Instructors were asked what paradigm was being taught in their introductory programming course, regardless of what is traditionally thought to apply to the language(s) in use. This question, understandably, caused some dissatisfaction in the comments section, with many participants noting that more than paradigm is taught in their course. Although this was to be expected, we wanted to be able to directly compare our results to \cite{mason+cooper:2014}, and so did not alter the question.

The results of this question are given in Table~\ref{tab:paradigm}. We can see that the most popular paradigm is object-oriented, followed by procedural and functional (logical was also offered as a choice but was not selected). 



%\subsubsection{Paradigm by nation and tariff score}
The results of the previous question were used to analyse the prevalence of paradigms across nations and tariff score groups. This analysis is displayed in Figure~\ref{fig:paradigmTariff}. Caution must be applied when interpreting these results, as participants could only choose one paradigm, even though more may be in use.

{\bf{[@James: Need to add some commentary here.]}}

\begin{figure}[ht]\vskip-12pt
\begin{center}
\subimport{plots/}{paradigmByTariffPercent.tex}
\end{center}\vskip-18pt
\caption{The breakdown of the main paradigm in use for every Tariff Group.\label{fig:paradigmTariff}}
\end{figure}

\begin{figure}
\begin{center}
\subimport{plots/}{ParadigmByLangPercent.tex}
\end{center}\vskip-18pt
\caption{The breakdown of the main paradigm in use for each programming language.\label{fig:paradigmLang}}
\end{figure}

%\subsubsection{Paradigm by language}

In the same way as above, the languages chosen were analysed with regard to the main paradigm in use. The results are given in Figure~\ref{fig:paradigmLang}. Again, caution must be applied, as for a given course, only one paradigm is chosen, even though more than one language and/or paradigm may be in use.

{\bf{[@James: Need to add some commentary here.]}}


%\begin{figure}
%\begin{center}
%\subimport{plots/}{TariffByParadigmPercent.tex}
%\end{center}
%\caption{The breakdown of Tariff Groups for each paradigm.}
%\end{figure}
%
%\begin{figure}
%\begin{center}
%\subimport{plots/}{langByParadigmPercent.tex}
%\end{center}\vskip-18pt
%\caption{The breakdown of programming languages in use for each paradigm.}
%\end{figure}







\subsection{Instructor Experience}
Participants were asked: ``How many years have you been involved in
teaching of introductory programming?''. The results, shown in
Table~\ref{tab:yearsTeaching}, indicate that of the survey
participants, the average was between 10 - 20 years.

{\bf{[Anything else to be said here?]}}

\begin{table}[ht]
\centering
\caption{The number of years the instructor has been involved in teaching introductory programming.}
\label{tab:yearsTeaching}
\begin{tabular}{ccccccc}
\hline
Years       & \textless 2 & 2 - 5 & 5 - 10 & 10 - 20 & 20 - 30 & \textgreater 30 \\ \hline
Instructors & 3          & 9     & 9      & 27      & 19      & 7              \\ \hline
\end{tabular}
\end{table}




\begin{figure}[ht]
\begin{center}
\subimport{plots/}{toolPercentCompare.tex}
\caption{Tool or environment popularity by percentage of courses and students.\label{fig:tools}}
\end{center}
\end{figure}


\subsection{IDEs and Tools}
\subsubsection{Choice of tool/IDE}
Participants in the survey were asked if they encouraged students in the first programming course to use environments and/or tools beyond simple text editors and command line compilers. The majority of participants of this question (74.4\% of 78 instructors) responded that they did encourage tools. Of the 58 instructors that did select a tool/IDE, the majority (59.6\%) use only one. See Table~\ref{tab:numTools} for more details on the number of tools and IDEs used.

\begin{table}[]
\centering
\caption{The number of tools/environments used in first programming courses.\ref{tab:numTools}}
\label{tab:numTools}
\begin{tabular}{cccccc}
\hline
Tools   & 1  & 2  & 3 & 4 & 8 \\ \hline
Courses & 34 & 15 & 6 & 2 & 1 \\ \hline
\end{tabular}
\end{table}

The survey asked participants to select the tools and IDEs in use in their introductory programming course out of a list of 24, which the option to specify ``Other''. Of the 24 provided, 12 were chosen at least once. The relative popularity of IDEs and tools is shown in Figure~\ref{fig:tools}. The tools and IDEs not selected at all were: AdaCore, Alice, App, Browser, Greenfoot, Jeroo, Jython, KTechLab, MySQL, Pelles, Quincy, Wing101 and Xcode.

The most popular tool/IDE in the survey was Eclipse, reported in 37.5\% of courses and scoring 30.6\% of tool/IDE instances, and 32.1\% when weighted by the number of students that use one or more tool/IDE. This is followed by BlueJ, which was reported in \% of courses and scored 18.4\% of tool/IDE instances, and 18.6\% when weighted by students. 

\subsubsection{Reasons for choice of tool/IDE}
Participants were also asked why each tool/IDE was chosen for their course, and asked to select from a list of reasons. The results of this are give in Figure~\ref{fig:reasonsTools}, for all tools and IDEs grouped together, and for the two most popular choices, Eclipse and BlueJ. 

\begin{figure}[ht]
\begin{center}
\subimport{plots/}{reasonsByCourseCompareTool.tex}
\end{center}
\caption{Reasons given for choosing a tool or environment by percentage for: all tools and environments; BlueJ; and Eclipse.\label{fig:reasonsTools}}
\end{figure}

\subsubsection{Reuse of tool/IDE}
Instructors were also asked whether the tool/IDE was used for an initial part of the first programming course or throughout the whole of the course; and whether it was used in any other course in the degree (Figure~\ref{fig:toolreuse}). 
\begin{figure}
\begin{center}
\subimport{plots/}{timingOtherCourseTool.tex}
\end{center}\vskip-18pt
\caption{For each tool or environment, whether it is used: for an initial part of the first programming course; throughout the whole of the first programming course; in any other course in the degree.\label{fig:toolreuse}}
\end{figure}

\subsubsection{Difficulty of tool/IDE}
In addition to this, instructors were asked to rate how difficult {\emph{they}} found the tool/IDE on a Likert scale from extremely easy (1) to extremely difficult (7), and also how difficult they believed {\emph{the students}} found the tool/IDE, shown in Figure~\ref{fig:toolhard}.

\begin{figure}
\begin{center}
\subimport{plots/}{DifficultyYouStudentsCompareTools.tex}
\end{center}\vskip-18pt
\caption{The median difficulty rating of tool/environment for the instructor and students to use, where 1 is `extremely easy' and 7 is `extremely difficult'.  Answers must have been given by at least two instructors.\label{fig:toolhard}}
\end{figure}

We note that, while Eclipse is the most popular tool by some way, it is also deemed to be most difficult. This, apparently perverse, practice might be explained by the extent of re-use of Eclipse in other courses.

%{\bf{[@James: Any other commentary on tools/IDEs?]}}


\subsection{Other Aspects of the Course}
\iffalse
\subsubsection{External Delivery}
\begin{figure}
\begin{center}
\subimport{plots/}{Steps.tex}
\end{center}\vskip-18pt
\caption{Steps taken to determine whether students have received unauthorised assistance on assignments.\label{fig:Plagiarise}}
\end{figure}
%\par
\subsubsection{Resources provided to students}
\fi
The questionnaire asked about the resources in terms of examples,
books etc. provided to students. The results are rather similar to
\cite[Figure 14]{mason+cooper:2014} so we do not repeat that here:
details are in the full paper.
%\par

%\subsubsection{Unauthorised assistance}
We  asked participants: ``what steps do you take to try to determine whether students have received unauthorised assistance on assignments?''. The details are in the full paper, but range from ``notice unlikely similarities'' (50\% of instructors) to ``interview some students/groups at random'', selected by only 5 instructors. {\bf @Ellen: is this right}

\iffalse
% {\bf{[@James: this was one of the questions given to me by Raina, but it's possible that it was only added to the list after [8] was published. So might not be correct to say that it was asked by them.]}} 
We
report our results in Figure \ref{fig:Plagiarise}, as we think they
are of general interest. 

%\begin{figure}
%\begin{center}
%\subimport{plots/}{Resources.tex}
%\end{center}\vskip-18pt
%\caption{Resources provided to students.}
%\end{figure}
\fi
\subsection{Aims of an Introductory Programming Course}

 \cite{mason+cooper:2014} asked their respondents for the aims of
their introductory programming course. They, and we, asked for (up to)
three aims. The authors then attempted to classify the free-text
answers into the same categorisation as \cite{mason+cooper:2014}
used. While it is trivial to map the written aim ``Thinking
algorithmically'' to \cite{mason+cooper:2014}'s ``Algorithmic
thinking'' and so on, many were not so clear: for example, we mapped
``To learn a specific language'' to ``syntax/writing basic
code''. There were also a class of aims, such as ``Establish
professional software development practices'', that seemed very
coherent, but didn't map clearly to the \cite{mason+cooper:2014}
aims. These we have categorised as ``Software Engineering''. Results of this question are shown in Figure~\ref{fig:aims}.

\begin{figure}
\begin{center}
\subimport{plots/}{Aims.tex}
\end{center}\vskip-18pt
\caption{Aims of the introductory course. \label{fig:aims}}
\end{figure}

\section{General Discussion}\label{discussion}

\subsection{The U.K. context}

\subsection{Comparison with Australasia}

Here we compare with \cite{mason+cooper:2014}, the latest Australasian
survey. We have already commented on the major difference in language
choice, which colours many of the other comparisons. n fact, the
U.K.'s language choices seem more similar to Australasia's 2010
choices \cite{mason-et-al:2012} and \cite[Table 4]{mason+cooper:2014}
than even Australasia's 2013 choices. It is hard to know which comes
first, but we also notice that our difficulty/utility data (Figure
\ref{fig:utility}) is somewhat different from \cite[Figures
7,8]{mason+cooper:2014}

Another difference shows up in the tools/enviroments are: Figure
\ref{fig:tools} versus \cite{mason+cooper:2014}'s Figure 11. There,
``None'' and ``Other'' were the top two categories, with Idle, at
15\%, the most popular named product. In the UK, ``None'' is second, ``Other'' is fifth
and Idle seventh.
%, with no ``None'' @Ellen: is that right?{\bf{[@Tom/James?: No, ``none'' scores second for us 28.2\% but I didn't think to include it in graph. I can add it if you like.]  @Ellen, yes please}}.


\section{Acknowledgements}

The authors would like to thank the participants for their engagement
with the survey, as well as the authors of [8] for providing us with
their survey and permission to use it.
% HEFCE? I think we should in the final version, but leave them out for now.
% CPHC? Not really
% GW4 - JHD has asked what the form of words should be.
We are grateful to the GW4 Alliance (Universities of Bath, Bristol,
Cardiff and Exeter) for funding the underpinning survey.
% bib
\bibliographystyle{abbrv}
\bibliography{sigcse2017}

\end{document}